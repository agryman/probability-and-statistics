\documentclass[11pt]{amsart}
\usepackage{geometry}                % See geometry.pdf to learn the layout options. There are lots.
\geometry{letterpaper}                   % ... or a4paper or a5paper or ... 
%\geometry{landscape}                % Activate for for rotated page geometry
\usepackage[parfill]{parskip}    % Activate to begin paragraphs with an empty line rather than an indent
\usepackage{graphicx}
\usepackage{amssymb}
\usepackage{epstopdf}
\DeclareGraphicsRule{.tif}{png}{.png}{`convert #1 `dirname #1`/`basename #1 .tif`.png}

\title{Probability and Statistics}
\author{Arthur Ryman}
%\date{}                                           % Activate to display a given date or no date

\begin{document}
\maketitle
\section{Introduction}

This article presents the basic concepts of probability and statistics from a mathematical point of view.
The approach here differs from the usual approach found in most textbooks in that it raises the level
of abstraction by viewing the subject matter in the light of category theory. 
The hope is that this point of view simplifies the subject by providing a concise unifying conceptual framework.

The main concept for unifying the subject is that of a probability space. 
Probability spaces are the objects of a category and random variables are its morphisms.

As an example of the utility of these concepts, consider the notion of a p-value.
The standard definition of a p-value is a function that assigns to a measurement the probability of observing a measurement as extreme or more extreme than the given value.
This definition is somewhat hard to grasp at first.
In terms of category theory, a p-value is simply a morphism from a probability space to the uniform probability space on the unit interval.
Here the uniform probability space on the unit interval is a very important important object in the category of probability spaces, and p-values are simply morphisms to it.

The underlying thesis of this article is that every important concept in probability and statistics has a natural and simple interpretation in terms of  category theory.

\section{Measurable Spaces}

A probability spaces is a type of measurable space.
A measurable space $M =(X,A)$ is a set $X$ together with a distinguished set $A$ of subsets that are capable of being measured.
The measurable sets are required to form a $\sigma$-algebra under the usual operators of set theory.

The axioms for a $\sigma$-algebra of subsets of $X$ are as follows:
\begin{itemize}
\item The empty set is measurable: $\emptyset \in A$
\item $X$ is measurable: $X \in A$
\item The complement of a measurable set is measurable: $\forall Y \in A, X \setminus Y \in A$
\item The union of a countable number of measurable sets is measurable: $\forall f: N -> A, \cup_{i \in N} f(i) \in A$
\end{itemize}

The axiom about the union of a countable number of measurable sets being measurable is related to the definition of the sum of infinite series.

A consequence of these axioms is that the intersection of a countable number of measurable sets is also measurable.

Clearly, if $X$ is any set then its power set $2^X$ is a $\sigma$-algebra.

If A and B are sigma algebras on X, then so their intersection $A \cap B$.
Given any collection C of subsets of X that is contained in the sigma algebras A and B, then C is also contained in their intersection.
Therefore, the intersection of all sigma algebras that contain C is also a sigma-algebra that contains C. It is the sigma-algebra generated by C.

When dealing with the real numbers, we are interested in the sigma algebra generated by the set of all closed, bounded intervals. 

A measure space is a measurable space together with a measure. 
A measure if a function $m: A -> R^{+}$ that assigns to any measurable set, a non-negative real number, where we have extended the real numbers with positive infinity and defined addition in the obvious way.

A measure space satisfies the following axioms:

\begin{itemize}
\item The empty set has measure 0: $m(\emptyset) = 0$
\item The measure of the union of a countable sequence of disjoint measurable sets is the sum of the measures of the sets in the sequence: $\forall f: N -> A | \forall i, j : N | i \neq j => A_i \cap A_j = \emptyset, m(\cup_i A_i) = \Sigma_i m(A_i)$
\end{itemize}

For example, if X is a countable set then m(Y) = #Y is a measure.




%\subsection{}



\end{document}  