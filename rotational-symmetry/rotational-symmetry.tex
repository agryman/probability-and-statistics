\documentclass[11pt]{amsart}
\usepackage{geometry}                % See geometry.pdf to learn the layout options. There are lots.
\geometry{letterpaper}                   % ... or a4paper or a5paper or ... 
%\geometry{landscape}                % Activate for for rotated page geometry
%\usepackage[parfill]{parskip}    % Activate to begin paragraphs with an empty line rather than an indent
\usepackage{graphicx}
\usepackage{amssymb}
\usepackage{epstopdf}
\DeclareGraphicsRule{.tif}{png}{.png}{`convert #1 `dirname #1`/`basename #1 .tif`.png}

\usepackage{amsthm}
\newtheorem{theorem}{Theorem}

\title{Rotational Symmetry}
\author{Arthur Ryman}
\date{\today}                                           % Activate to display a given date or no date

\begin{document}
\maketitle
\section{Introduction}

I recently needed to generate a random sample from the uniform distribution on the sphere $S^n$ where $n$ was very large.
My strategy was to generate a rotationally symmetric sample of non-zero vectors $v$ in some region of $\mathbb{R}^{n+1}$
and then normalize those to get unit vectors $u = v/\|v\|$ on $S^n$.
I first attempted to generate a random sample from the uniform distribution on the cube $[-1,1]^n$ and then take the subset of those
vectors $v$ such that $0 < \|v\| \leq 1$.
However, it turns out that the volume enclosed by $S^n$ becomes a very small fraction of the volume of $[-1,1]^n$ as $n$ gets large.
The probability that $\|v\| \leq 1$ goes to zero, so this approach fails.

I then remembered the well-known fact that the distribution of $n$ independent and identically  distributed (IID) standard normal variates is rotationally symmetric.
This symmetry seems miraculous and in fact only occurs for the normal distribution.
This article gives a proof of that fact.

\section{Rotational Symmetry of $2$ IID Variates}

Let the usual polar coordinate system on the plane be:
$$
\begin{aligned}
x &= r \cos \theta \\
y &= r \sin \theta
\end{aligned}
$$

Note that:
$$
\begin{aligned}
\frac{\partial}{\partial\theta}	&= \frac{\partial x}{\partial\theta} \frac{\partial}{\partial x} + \frac{\partial y}{\partial\theta} \frac{\partial}{\partial y} \\
						&= -r \sin\theta \frac{\partial}{\partial x} + r \cos\theta \frac{\partial}{\partial y} \\
						&= -y\frac{\partial}{\partial x} + x\frac{\partial}{\partial y} 
\end{aligned}
$$

A function $F$ on the plane is rotationally symmetric when it is independent of $\theta$, i.e.:
$$
\frac{\partial F}{\partial \theta} = 0
$$
or, in terms of $x$ and $y$:
$$
-y\frac{\partial F}{\partial x} + x\frac{\partial F}{\partial y} = 0
$$
Therefore, a rotationally symmetric function must satisfy the following linear partial differential equation:
$$
x\frac{\partial F}{\partial y} = y\frac{\partial F}{\partial x}
$$

The probability density function $f$ of a normal distribution with mean zero and variance $\sigma^2$ is
$$
f(x) = Ce^{-\frac{x^2}{2\sigma^2}}
$$
where the normalization factor is $C=1/\sqrt{2\pi}$.

\begin{theorem}
The joint distribution of two IID random variates is rotationally symmetric if and only the random variates are normally distributed with mean zero.
\end{theorem}

\begin{proof}

First prove that the distribution of two IID normal variates with mean zero is rotationally symmetric.
The probability density function $F$ of the joint distribution is:
$$
\begin{aligned}
F(x,y)	&= f(x)f(y) \\
		&= Ce^{-\frac{x^2}{2\sigma^2}}Ce^{-\frac{y^2}{2\sigma^2}} \\
		&= C^2e^{-\frac{x^2}{2\sigma^2}-\frac{y^2}{2\sigma^2}} \\
		&= C^2 e^{-\frac{x^2+y^2}{2\sigma^2}} \\
		&= C^2 e^{-\frac{r^2}{2\sigma^2}}
\end{aligned}
$$

which does not depend on $\theta$.
Therefore the joint distribution is rotationally symmetric.

Now prove that if the joint distribution of two IID variates is rotationally symmetric then the variates are normally distributed with mean zero.
Let $g$ be the probability density function of any random variate such that the joint distribution
$$G(x,y) = g(x)g(y)$$ 
is rotationally symmetric.
The partial derivatives of $G$ are:
$$
\begin{aligned}
\frac{\partial G}{\partial x} &= g'(x)g(y) \\
\frac{\partial G}{\partial y} &= g(x)g'(y)
\end{aligned}
$$
The condition for rotational symmetry is:
$$
x\frac{\partial G}{\partial y} = y\frac{\partial G}{\partial x}
$$
Substituting in the expressions for the partial derivatives of $G$ we have:
$$
x g(x)g'(y) = y g'(x)g(y)
$$
Dividing both sides by $x y g(x) g(y)$ we have:
$$
\frac{g'(x)}{x g(x)} = \frac{g'(y)}{y g(y)}
$$
The LHS depends only on $x$ and the RHS depends only on $y$.
But $x$ and $y$ can vary independently.
Therefore each side of the equation must be equal to some constant, say $\alpha$:
$$
\frac{g'(x)}{x g(x)} = \alpha
$$
Using the identity $(\ln g)'(x) = g'(x)/g(x)$ we can rewrite the preceeding equation as:
$$
(\ln g)'(x) = \alpha x
$$
Integrating, we get:
$$
\ln g(x) = \frac{\alpha}{2} x^2 + \beta
$$
where $\beta$ is a constant of integration.
Exponentiating, we get:
$$
\begin{aligned}
g(x) 	&= e^{\frac{\alpha}{2} x^2 + \beta} \\
	&= B e^{\frac{\alpha}{2} x^2}
\end{aligned}
$$
where $B = e^\beta$.
Since $g$ is a probability density function we must have
$$
\lim_{|x|\rightarrow\infty}g(x) = 0
$$
so $\alpha$ must be negative.

Therefore setting
$$
\begin{aligned}
\alpha 	&= -\frac{1}{\sigma^2} \\
B 		&= C
\end{aligned}
$$
we see that $g$ is the probability density function of a normal distribution with mean zero and variance $\sigma^2$.
\end{proof}

\section{Rotational Symmetry of $n$ IID Variates}

\begin{theorem}
The joint distribution of $n$ IID random variates is rotationally symmetric if and only the random variates are normally distributed with mean zero.
\end{theorem}

\begin{proof}
Clearly, the joint distribution of $n$ IID normal variates with mean zero is rotationally symmetric.

To prove the converse it suffices to consider rotations in the plane defined by any pair of coordinates. 
The preceding theorem implies that the variates are normal with mean zero.
\end{proof}

\end{document}  